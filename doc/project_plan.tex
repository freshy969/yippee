\documentclass[11pt, letterpaper, oneside, twocolumn]{article}
\usepackage{ amssymb }
\usepackage{fancyvrb}
\usepackage{float}
\usepackage{subfig}
\usepackage{graphicx,dblfloatfix}
\usepackage[hmargin=1.25cm, vmargin=1.5cm]{geometry}               
\usepackage{enumitem}
\setitemize{noitemsep,topsep=0pt,parsep=0pt,partopsep=0pt}
\DefineVerbatimEnvironment{code}{Verbatim}{fontsize=\small}

\newcommand{\tab}{\hspace*{2em}}
\floatstyle{plain}
\restylefloat{figure}
\begin{document}
\title{Yippee: Web Search for the New Millenium}
\author{	TJ Du
	\and Chris Imbriano
	\and Margarita Miranda
	\and Nikos Vasilakis}
\date{April 2011}

\maketitle

\section{ Introduction }

To say the world wide web is enormous would be an understatement.  There are approximately 

\section{ High-level Approach }

The Yippee Search Engine will be comprised of 4 main components: a web crawler to recursively download pages from the web, an indexer to index crawled web pages, an implementation of 
\begin{itemize}
\item Web-crawler that will recursively download pages from the web
\item Indexer/TDF/IDF Retrieval Engine that will index relevant document features for later retrieval 
\item PageRank Calculations based on the Google PageRank description REFERENCE
\item Search Engine and User Interface
\end{itemize}

Most or all of the functionality of these components will be distributed among a number of nodes coordinated with the Pastry substrate. 

The architecture of the Search Engine is shown in figure

\section{ Milestones }

\label{sec:SOAR} %software architecture
\begin{figure}[!b]
  \centering
  \includegraphics[scale=0.50]{figures/yippee_map.pdf}
  \caption{Software Architecture.}
\end{figure}



\section{ Project Goals }

Our main objective in this project is to create a stable and efficient search engine that quickly evaluates queries with high accuracy. We aim to maximize the correlation between our result rankings and those of other major search engines. Additionally we strive to make our indexing resilient to crashes, distribute PageRank calculation, and to interleave external search results into our own. If we have additional time, we will implement a spell-check and AJAX support for users to give feedback on query results (along with appropriate weighting for result re-ranking).

We will 

\subsection{ High Accuracy }
\subsection{ Crawler Efficiency }
\subsection{ Query Efficiency }

\section{ Milestones }

\subsection{Milestone 1: 3/20-4/8}
First, we plan to create a complete map of our project architecture to ensure we
have a thoughtful design. Then, we will deploy minimally functioning
Pastry nodes locally and write the scripts to do so. Once that is
working, we will migrate those nodes to EC2. The Crawler will have a basic
implementation of pluggable components. The Indexer will have a static lexicon
and all major components communicating with each other. 


\subsection{Milestone 2: 4/9-4/15}
We will complete both the Crawler and Indexer and start these components on EC2
to begin building the corpus and index. 


\subsection{Milestone 3: 4/16-4/22}
A basic PageRank Module should be created and working. We will create a wire
frame for the UI of our web application. For our final report, we will write a complete outline with finished report sections for the crawler
and indexer. 


\subsection{Milestone 4: 4/22-4/30}
We will evaluate the performance of our PageRank and finish polishing the UI. 


\section{ Division of Labor }

While each member of the group is ultimately responsible for monitoring progress and completion of one of the four components described in Section 2, no individual is tasked with its implementation.  Instead, the group collectively determines the high level architecture and component interfaces, then a pair of members implements their assigned component.  The other two group members will write black box tests of the agreed upon interface without inspecting the source code such that their tests are unbiased towards any particular implementation decision. The sign-off responsibility is divided as follows: Crawler - Nikos, Indexer - Margarita, PageRank - Chris, Search Engine and Web UI - TJ.

For the first milestone, Chris and Nikos will work on the Crawler and the preliminary components of FreePastry, and TJ and Margarita will work on the Indexer. The same approach will be used for PageRank and User Interface but the team members may be shuffled.


\section{References}

\begin{enumerate}
\item Reference 1
\item Reference 2
\item Reference 3
\end{enumerate}

\end{document}
